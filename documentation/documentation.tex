\documentclass[a4paper]{article}

\usepackage[italian]{babel}
\usepackage{float}
\usepackage{makecell}

\usepackage{hyperref}
\hypersetup{
  colorlinks=false,
}

% Code blocks
\usepackage{listings}
\usepackage{xcolor}
\definecolor{codegreen}{rgb}{0,0.6,0}
\definecolor{codegray}{rgb}{0.5,0.5,0.5}
\definecolor{codepurple}{rgb}{0.58,0,0.82}
\definecolor{backcolour}{rgb}{0.95,0.95,0.95}

\lstdefinestyle{mystyle}{
	backgroundcolor=\color{backcolour},
	commentstyle=\color{codegreen},
	keywordstyle=\color{magenta},
	numberstyle=\tiny\color{codegray},
	stringstyle=\color{codepurple},
	basicstyle=\ttfamily\footnotesize,
	breakatwhitespace=false,
	breaklines=true,
	captionpos=b,
	keepspaces=true,
	numbers=left,
	numbersep=5pt,
	showspaces=false,
	showstringspaces=false,
	showtabs=false,
	tabsize=2
}

\lstset{style=mystyle}

\begin{document}

\begin{titlepage}
	\begin{center}
		\vspace*{1cm}

		\Huge
		\textbf{Aggiornamenti OTA con Secure-Boot e Flash encryption}

		\vspace{0.5cm}
		\LARGE
		UniVR - Dipartimento di Informatica

		\vspace{1.5cm}

		\textbf{Fabio Irimie}

		\vfill


		\vspace{0.8cm}


		Tesi di laurea 2025/2026

	\end{center}
\end{titlepage}


\tableofcontents
\pagebreak

\section{Introduzione}
Questo progetto consiste nell'implementazione di un sistema che permetta di aggiornare
il firmware dell'ESP32 da remoto (Over The Air) tramite Wi-Fi. L'obiettivo principale è
quello di attivare le funzionalità di sicurezza del microcontrollore in modo da proteggere
il dispositivo da accessi non autorizzati. Le funzionalità di sicurezza includono:
\begin{itemize}
  \item \textbf{Secure OTA}: Garantisce che il nuovo firmware sia autentico e non
    compromesso
  \item \textbf{Secure Boot}: Impedisce l'esecuzione di firmware non autorizzato
  \item \textbf{Flash Encryption}: Protegge i dati memorizzati nella memoria flash
    del dispositivo
\end{itemize}


\section{Cenni teorici}

\subsection{Aggiornamenti OTA (Over The Air)}
Gli aggiornamenti OTA permettono di aggiornare il firmware del dispositivo durante la
sua normale esecuzione, senza la necessità di collegarlo fisicamente a un computer. Le
modalità di aggiornamento si distinguono in base alla vulnerabilità del sistema:
\begin{itemize}
  \item \textbf{Modalità sicura}: L'aggiornamento di alcune partizioni è resiliente,
    cioè garantisce l'operabilità del dispositivo anche in caso di perdita di alimentazione
    o di errore durante l'aggiornamento. Solo il seguente tipo di partizione supporta
    la modalità sicura:
    \begin{itemize}
      \item \textbf{Application}: OTA configura la partition table in modo da avere
        due partizioni per l'aggiornamento (\texttt{ota\_0} e \texttt{ota\_1}) e una
        partizione per lo stato di boot (\texttt{ota\_data}). Durante l'aggiornamento il
        nuovo firmware viene scritto nella partizione OTA attualmente non selezionata
        per il boot. Una volta completato l'aggiornamento, la partizione \texttt{ota\_data}
        viene aggiornata per indicare che la partizione OTA appena scritta deve essere
        utilizzata al boot successivo. Se la partizione \texttt{ota\_data} non contiene
        alcun dato il dispositivo esegue il boot dalla partizione \texttt{factory}.

        La partition table con due partizioni OTA è la seguente:
        \begin{lstlisting}
# ESP-IDF Partition Table
# Name,   Type, SubType, Offset,  Size, Flags
nvs,      data, nvs,     0x9000,  0x4000,
otadata,  data, ota,     0xd000,  0x2000,
phy_init, data, phy,     0xf000,  0x1000,
factory,  app,  factory, 0x10000,  1M,
ota_0,    app,  ota_0,   0x110000, 1M,
ota_1,    app,  ota_1,   0x210000, 1M,
        \end{lstlisting}
    \end{itemize}
  \item \textbf{Modalità non sicura}: L'aggiornamento di alcune partizioni è vulnerabile,
    cioè in caso di perdita di alimentazione o di errore durante l'aggiornamento il
    dispositivo potrebbe non essere più operabile. Una partizione temporanea riceve
    i dati della nuova immagine e, una volta completato il trasferimento, l'immagine viene
    copiata nella partizione di destinazione. Se l'operazione di copia viene interrotta
    potrebbero verificarsi problemi di boot. Le partizioni che supportano la modalità non
    sicura sono:
    \begin{itemize}
      \item \textbf{Bootloader}
      \item \textbf{Partition Table}
      \item \textbf{Partizioni data} (ad esempio NVS, FAT, ecc...)
    \end{itemize}
\end{itemize}

\subsubsection{Partizione OTA Data}
Al primo avvio del dispositivo la partizione \texttt{ota\_data} deve essere vuota (tutti i
byte a 0xFF) in modo da far eseguire il boot dal'applicazione nella partizione
\texttt{factory}. Se l'applicazione in \texttt{factory} non è presente viene eseguito il boot
della prima partizione OTA disponibile (di solito \texttt{ota\_0}).

Dopo il primo aggiornamento OTA, la partizione \texttt{ota\_data} viene aggiornata per
indicare quale partizione OTA deve essere utilizzata al successivo boot. La dimensione
di \texttt{ota\_data} è di due settori (0x2000 bytes = 8192 bytes) in modo da evitare
errori mentre si scrive la partizione. I due settori sono cancellati indipendentemente
e scritti con gli stessi dati. In questo modo se i dati dei due settori non coincidono
viene usato un counter per determinare quale settore è stato scritto più recentemente.

\subsubsection{App rollback}
L'obiettivo dell'app rollback è quello di tenere il funzionante il dispositivo dopo un
aggiornamento e permette di tornare alla versione precedente del firmware se la nuova
versione non funziona correttamente (solo le partizioni OTA possono effettuare il rollback).
Dopo un aggiornamento OTA con rollback attivo si hanno le seguenti possibilità:
\begin{itemize}
  \item Se l'app funziona bene \texttt{esp\_ota\_mark\_app\_valid\_cancel\_rollback()}
    imposta lo stato dell'applicazione a \texttt{ESP\_OTA\_IMG\_VALID}.

  \item Se l'app non funziona correttamente il dispositivo esegue il rollback alla versione
    precedente e \texttt{esp\_ota\_mark\_app\_invalid\_rollback()} imposta lo stato
    dell'applicazione a \texttt{ESP\_OTA\_IMG\_INVALID}.

  \item Se l'impostazione \texttt{CONFIG\_BOOTLOADER\_APP\_ROLLBACK\_ENABLE} è abilitata
    e viene effettuato un reset, allora viene effettuato un rollback senza chiamare
    nessuna funzione nell'applicazione. Questa opzione permette di intercettare la
    prima esecuzione di una nuova applicazione per confermare che funzioni correttamente.
\end{itemize}

\vspace{1em}
\noindent
Gli stati che controllano il processo di selezione dell'applicazione sono:
\begin{table}[H]
  \centering
  \begin{tabular}{|l|l|}
    \hline
    \textbf{Stato} & \textbf{Restrizioni sulla nuova app} \\
    \hline
    \texttt{ESP\_OTA\_IMG\_VALID} & Nessuna restrizione. Verrà selezionata \\
    \hline
    \texttt{ESP\_OTA\_IMG\_UNDEFINED} & Nessuna restrizione. Verrà selezionata \\
    \hline
    \texttt{ESP\_OTA\_IMG\_INVALID} & Non verrà selezionata \\
    \hline
    \texttt{ESP\_OTA\_IMG\_ABORTED} & Non verrà selezionata \\
    \hline
    \texttt{ESP\_OTA\_IMG\_NEW} & \makecell[l]{Se l'opzione \\
    \texttt{CONFIG\_BOOTLOADER\_APP\_ROLLBACK\_ENABLE} \\
    è abilitata, l'app verrà selezionata solo una \\
    volta. Nel bootloader lo stato viene subito \\
    impostato a \texttt{ESP\_OTA\_IMG\_PENDING\_VERIFY}.} \\
    \hline
      \texttt{ESP\_OTA\_IMG\_PENDING\_VERIFY} & \makecell[l]{Se l'opzione \\
    \texttt{CONFIG\_BOOTLOADER\_APP\_ROLLBACK\_ENABLE} \\
    è abilitata, l'app non verrà selezionata. \\
    Nel bootloader lo stato viene \\
    impostato a \\
    \texttt{ESP\_OTA\_IMG\_ABORTED}.} \\
    \hline
  \end{tabular}
\end{table}
\noindent
L'impostazione di questi stati avviene nei seguenti casi:
\begin{itemize}
  \item \texttt{ESP\_OTA\_IMG\_VALID}: impostato dalla funzione \\
    \texttt{esp\_ota\_mark\_app\_valid\_cancel\_rollback()}.

  \item \texttt{ESP\_OTA\_IMG\_UNDEFINED}: impostato dalla funzione \\
    \texttt{esp\_ota\_set\_boot\_partition()} se l'impostazione \\
    \texttt{CONFIG\_BOOTLOADER\_APP\_ROLLBACK\_ENABLE} è disabilitata.

  \item \texttt{ESP\_OTA\_IMG\_NEW}: impostato dalla funzione \\
    \texttt{esp\_ota\_set\_boot\_partition()} se l'impostazione \\
    \texttt{CONFIG\_BOOTLOADER\_APP\_ROLLBACK\_ENABLE} è abilitata.

  \item \texttt{ESP\_OTA\_IMG\_INVALID}: impostato dalla funzione \\
    \texttt{esp\_ota\_mark\_app\_invalid\_rollback()} o \\
    \texttt{esp\_ota\_mark\_app\_invalid\_rollback\_and\_reboot()}.

  \item \texttt{ESP\_OTA\_IMG\_ABORTED}: impostato se l'operabilità dell'applicazione non
    è stata confermata e avviene un reboot quando l'impostazione \\
    \texttt{CONFIG\_BOOTLOADER\_APP\_ROLLBACK\_ENABLE} è abilitata.

  \item \texttt{ESP\_OTA\_IMG\_PENDING\_VERIFY}: impostato nel bootloader se l'impostazione \\
    \texttt{CONFIG\_BOOTLOADER\_APP\_ROLLBACK\_ENABLE} è abilitata e l'applicazione
    selezionata è nello stato \texttt{ESP\_OTA\_IMG\_NEW}.
\end{itemize}



% \section{Cenni teorici}
% \section{Implementazione}
% \section{Analisi dei risultati}
% \section{Conclusioni}

\end{document}
